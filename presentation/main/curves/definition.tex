\begin{frame}
  \[ y^2 = x^3+ax+b - \text{эллиптическая кривая} \]
  \[ x^2/a^2+y^2/b^2 = 1 - \text{эллипс} \]
  \centering  
  \begin{tikzpicture}
    \begin{axis}[
      width=5cm, % Ширина графика
      height=6cm, % Высота графика
      domain=-5:5, % Диапазон по оси X
      y domain=-5:5, % Диапазон по оси Y
      samples=50, % Количество точек для построения
      smooth, % Сглаживание графика
      grid=both, % Включаем сетку
      enlargelimits=true, % Расширяем границы графика
      view={0}{90}, % Вид сверху (2D)
      name=plot1 % Имя для первого графика
  ]
  \addplot3[
    contour gnuplot={
        levels={0},
        labels=false,
        draw color=RNDSblue,
        cmd={
          gnuplot -c ./graphs/elliptic_curve.gp "./tmp/main_contourtmp0.table"
        }
    },
    thick,
    RNDSblue,
    domain=-3:4,
    y domain=-9:9
] {y^2 - x^3 - 9};
 \legend{secp256k1} % Легенда
\end{axis}
  
    % Второй график: повёрнутый эллипс
    \begin{axis}[
      width=5cm, % Ширина графика
      height=6cm, % Высота графика
      domain=-5:5, % Диапазон по оси X
      y domain=-5:5, % Диапазон по оси Y
      samples=50, % Количество точек для построения
      smooth, % Сглаживание графика
      grid=both, % Включаем сетку
      enlargelimits=true, % Расширяем границы графика
      view={0}{90}, % Вид сверху (2D)
      at={(plot1.right of north east)}, % Позиция второго графика
      anchor=left of north west % Привязка
    ]
      % Рисуем повёрнутый эллипс
      \addplot[
        red,
        thick,
        domain=0:360,
        samples=100,
        variable=\t,
      ] ({cos(45) * 2*cos(t) + sin(45) * sin(t)}, {-sin(45) * 2*cos(t) + cos(45) * sin(t)});
      \legend{Это эллипс} % Легенда
    \end{axis}
  \end{tikzpicture}
  \end{frame}